\documentclass[twoside]{article}

\usepackage{lipsum} % Package to generate dummy text throughout this template

\usepackage[sc]{mathpazo} % Use the Palatino font
\usepackage[T1]{fontenc} % Use 8-bit encoding that has 256 glyphs
\linespread{1.05} % Line spacing - Palatino needs more space between lines
\usepackage{microtype} % Slightly tweak font spacing for aesthetics

\usepackage[hmarginratio=1:1,top=32mm,columnsep=20pt]{geometry} % Document margins
\usepackage{multicol} % Used for the two-column layout of the document
\usepackage[hang, small,labelfont=bf,up,textfont=it,up]{caption} % Custom captions under/above floats in tables or figures
\usepackage{booktabs} % Horizontal rules in tables
\usepackage{float} % Required for tables and figures in the multi-column environment - they need to be placed in specific locations with the [H] (e.g. \begin{table}[H])
\usepackage{hyperref} % For hyperlinks in the PDF

\usepackage{lettrine} % The lettrine is the first enlarged letter at the beginning of the text
\usepackage{paralist} % Used for the compactitem environment which makes bullet points with less space between them

\usepackage{braket}
\usepackage{array}
\usepackage{calc}
\usepackage{graphicx}
\usepackage{listings}
\usepackage{kotex}
%\usepackage[sorting=none]{biblatex}
%\addbibresource{ref.bib}

\lstset{frame=tb,
  language=lisp,
  aboveskip=3mm,
  belowskip=3mm,
  showstringspaces=false,
  columns=flexible,
  basicstyle={\small\ttfamily},
  numbers=none,
  numberstyle=\tiny\color{gray},
  keywordstyle=\color{blue},
  commentstyle=\color{dkgreen},
  stringstyle=\color{mauve},
  breaklines=true,
  breakatwhitespace=true
  tabsize=3}

\lstnewenvironment{Python}
  {\lstset{
  language=Python, 
}}
  {}

\lstnewenvironment{C}
  {\lstset{
  language=C, 
}}
  {}
  
\lstnewenvironment{Java}
  {\lstset{
  language=Java, 
}}
  {}
  
\lstnewenvironment{scheme}
  {\lstset{
  morekeywords={define-type, define, type-case, match}
}}
  {}
  
\lstnewenvironment{expr}
  {\lstset{
	morekeywords={with, deffun, fun}
}}
  {}

\usepackage{color}
\usepackage[table,xcdraw]{xcolor}
\usepackage{adjustbox}


\definecolor{dkgreen}{rgb}{0,0.6,0}
\definecolor{gray}{rgb}{0.5,0.5,0.5}
\definecolor{mauve}{rgb}{0.58,0,0.82}



\hypersetup{%
    pdfborder = {0 0 0}
}



\usepackage{abstract} % Allows abstract customization
\renewcommand{\abstractnamefont}{\normalfont\bfseries} % Set the "Abstract" text to bold
\renewcommand{\abstracttextfont}{\normalfont\small\itshape} % Set the abstract itself to small italic text

\usepackage{titlesec} % Allows customization of titles
%\renewcommand\thesection{\Roman{section}} % Roman numerals for the sections
\renewcommand\thesubsection{\Roman{subsection}} % Roman numerals for subsections
\titleformat{\section}[block]{\large\scshape\centering}{\thesection.}{1em}{} % Change the look of the section titles
\titleformat{\subsection}[block]{\large}{\thesubsection.}{1em}{} % Change the look of the section titles

\usepackage{fancyhdr} % Headers and footers
\pagestyle{fancy} % All pages have headers and footers
\fancyhead{} % Blank out the default header
\fancyfoot{} % Blank out the default footer
\fancyhead[C]{ NN101 Introduction to Neural Net} % Custom header text
\fancyfoot[RO,LE]{\thepage} % Custom footer text


%----------------------------------------------------------------------------------------
%	TITLE SECTION
%----------------------------------------------------------------------------------------

\begin{document}
\begin{titlepage}

\newcommand{\HRule}{\rule{\linewidth}{0.5mm}} % Defines a new command for the horizontal lines, change thickness here

\center % Center everything on the page
 
%----------------------------------------------------------------------------------------
%	HEADING SECTIONS
%----------------------------------------------------------------------------------------

\vspace*{3cm}
\textsc{\Large NN101}\\[0.5cm] % Major heading such as course name
\textsc{\large Introduction to Neural Net}\\[0.5cm] % Minor heading such as course title

%----------------------------------------------------------------------------------------
%	TITLE SECTION
%----------------------------------------------------------------------------------------

\HRule \\[0.4cm]
{ \huge \bfseries Hitchhiker's Guide to Neural Network}\\[0.4cm] % Title of your document
\HRule \\[1.5cm]
 
%----------------------------------------------------------------------------------------
%	AUTHOR SECTION
%----------------------------------------------------------------------------------------

\begin{minipage}{0.4\textwidth}
\begin{flushleft} \large
\emph{Author:}\\
Seungwoo \textsc{Schin} \\% Your name
\end{flushleft}
\end{minipage}
\begin{minipage}{0.4\textwidth}
\begin{flushright} \large
\emph{Typeset by:} \\
Seungwoo \textsc{Schin} % Supervisor's Name
\end{flushright}
\end{minipage}\\[4cm]

% If you don't want a supervisor, uncomment the two lines below and remove the section above
%\Large \emph{Author:}\\
%John \textsc{Smith}\\[3cm] % Your name

\textsc{ KAIST, School of Computing}\\[1.5cm] % Name of your university/college

%----------------------------------------------------------------------------------------
%	DATE SECTION
%----------------------------------------------------------------------------------------

{\large \today}\\[3cm] % Date, change the \today to a set date if you want to be precise
%2015 Spring Semester

%----------------------------------------------------------------------------------------
%	LOGO SECTION
%----------------------------------------------------------------------------------------

%\includegraphics{Logo}\\[1cm] % Include a department/university logo - this will require the graphicx package
 
%----------------------------------------------------------------------------------------

%\vfill % Fill the rest of the page with whitespace

\end{titlepage}

% Table of contents 

\tableofcontents
\newpage

\section{Introduction} 

This document contains theoretical aspects and actual implementation of various neural nets. 

Structure of this document is organized as follows: in section 1, general introduction of machine learning and neural net will be given. From section 2, each neural net will be studied in detail. Implementation by python and its test results will be given for each neural net. For now\footnote{2017.12.07}, architectures below will be covered in this document\footnote{List from \href{https://en.wikipedia.org/wiki/Types_of_artificial_neural_networks}{wikipedia} and \href{http://www.asimovinstitute.org/neural-network-zoo/}{Asimov Institute}}. 
The list below is sorted in timal order based on the publication date of original pdf, so that reader can follow the historic timeline of artificial neural net development. 

\begin{itemize} 
\item Perceptron/Feedforward Network \cite{perceptron} 
\item Kohonen Network(KN) \cite{KN}
\item Boltzmann Machine(BM) \cite{BM}
\item Restricted BM(RBM) \cite{RBM}
\item Radial Basis Network(RBF) \cite{RBF}
\item AutoEncoder(AE) \cite{AE}
\item Hopfield Network(HN) \cite{HN}
\item Recurrent Neural Network(RNN) \cite{RNN}
\item Support Vector Machine(SVM) \cite{SVM}
\item Long/Short Term Memory(LSTM) \cite{LSTM}
\item Bidirectional RNN(Bi-RNN) \cite{Bi-RNN}
\item Deep Convolutioanl Network(DCNN) \cite{DCNN}
\item Liquid State Machine(LSM) \cite{LSM}
\item Echo State Network(ESN) \cite{ESN}
\item Sparse AE(SAE) \cite{SAE}
\item Deep Belief Network(DBN) \cite{DBN}
\item Denoising AE(DAE) \cite{DAE}
\item Deconvolutional Network(DN) \cite{DN}
\item Variational AE(VAE) \cite{VAE}
\item Markov Chain(MC) \cite{MC}
\item Gated Recurrent Unit(GRU) \cite{GRU}
\item Generative Adversarial Network(GAN) \cite{GAN}
\item Neural Turing Machine(NTM) \cite{NTM}
\item Deep Convolutional Inverse Graphics Network(DCIGN) \cite{DCIGN}
\item Extreme Learning Machine(ELM) \cite{ELM}
\item Deep Residual Network(DRN) \cite{DRN}
\end{itemize}

Implementation for each structures will be uploaded on github\footnote{\href{https://github.com/principia12/NN101}{https://github.com/principia12/NN101}}.

After all implementations are covered, optimization of neural networks will be discussed in two perspectives: application of evolutionary algorithms on topology of neural net architecture and evolutionary algorithm on generatlization of activation function. 

\newpage
\section{Background for Machine Learning and Neural Network} 

\subsection{Introduction to Machine Learning} 

\subsection{Introduction to Neural Network}
\newpage
\section{Various Neural Networks} 
\subsection{Perceptron/Feedforward Network \cite{perceptron} }
\subsection{Kohonen Network(KN) \cite{KN}}
\subsection{Boltzmann Machine(BM) \cite{BM}}
\subsection{Restricted BM(RBM) \cite{RBM}}
\subsection{Radial Basis Network(RBF) \cite{RBF}}
\subsection{AutoEncoder(AE) \cite{AE}}
\subsection{Hopfield Network(HN) \cite{HN}}
\subsection{Recurrent Neural Network(RNN) \cite{RNN}}
\subsection{Support Vector Machine(SVM) \cite{SVM}}
\subsection{Long/Short Term Memory(LSTM) \cite{LSTM}}
\subsection{Bidirectional RNN(Bi-RNN) \cite{Bi-RNN}}
\subsection{Deep Convolutioanl Network(DCNN) \cite{DCNN}}
\subsection{Liquid State Machine(LSM) \cite{LSM}}
\subsection{Echo State Network(ESN) \cite{ESN}}
\subsection{Sparse AE(SAE) \cite{SAE}}
\subsection{Deep Belief Network(DBN) \cite{DBN}}
\subsection{Denoising AE(DAE) \cite{DAE}}
\subsection{Deconvolutional Network(DN) \cite{DN}}
\subsection{Variational AE(VAE) \cite{VAE}}
\subsection{Markov Chain(MC) \cite{MC}}
\subsection{Gated Recurrent Unit(GRU) \cite{GRU}}
\subsection{Generative Adversarial Network(GAN) \cite{GAN}}
\subsection{Neural Turing Machine(NTM) \cite{NTM}}
\subsection{Deep Convolutional Inverse Graphics Network(DCIGN) \cite{DCIGN}}
\subsection{Extreme Learning Machine(ELM) \cite{ELM}}
\subsection{Deep Residual Network(DRN) \cite{DRN}} 
\newpage
\section{Application of Evolution Algorithm on Neural Networks}

\subsection{Neural Network Architecture Generation via Evolution Algorithm}

\subsection{Activation Function Optimization via Evolution Algorithm}  
\newpage
%\printbibliography
\bibliographystyle{unsrt}
\bibliography{ref} 

\end{document}



























